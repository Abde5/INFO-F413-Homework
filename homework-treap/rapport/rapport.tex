\documentclass[a4paper,10pt]{article}
\usepackage[utf8]{inputenc}
\usepackage{listings}
\usepackage{color}
\usepackage{url}
\usepackage{hyperref}
\usepackage{graphicx}
\usepackage{pgf}
\usepackage{amsfonts}

\definecolor{grey}{rgb}{0.9,0.9,0.9}

\lstset{
language=Python,
basicstyle=\footnotesize\fontfamily{pcr},
backgroundcolor=\color{grey},
numbers=left,
numberstyle=\tiny,
numbersep=5pt,
showstringspaces=false,
tabsize=3,
breaklines=true
}

%opening
\title{Random Treaps}
\author{Abdeselam El-Haman Abdeselam}

\begin{document}
\sloppy
\maketitle

Treaps are interesting data structures that act as trees and heaps at the same time
with its 2 keys per node. A randomized implementation of these structures are very appealing
in the focus of this course. In this document an analysis is done, showing that some properties
about tree balance are true and it helps to prove that in expectance, a random treap
is a well-balanced binary search tree.

\section{Treaps vs Quicksort}

The \textit{memoryless} property of a tree shows us that, as a priority for a node is chosen
independently for every node, we would know the structure of the tree if we knew those
random values before inserting them.

This property makes us able to make a parallelism with the behaviour of quicksort with 2
important characteristics of random treaps, the priority and its binary search similarity.
The priority of a node will position the node
up on the tree if it's higher, or down if it's lower.

If we take a treap, we'll see that every element in the node has been compared to the root ($x$) once,
and the root is the element with the highest priority. Then, the $x$ node has a sub-tree as his
left child and another one as his right child. We can see then that the elements of the left sub-tree
have been compared with the root of that sub-tree, but never been compared with the trees in the other
subtree. If we iterate like this, the root of every sub-tree in the treap, or the node at the $i$th level, can be seen as the \textit{pivot} chosen in the $i$th iteration of the quicksort algorithm. The
higher the random value of the priority is, the sooner it'd be appear in quicksort as a pivot. And the
comparaisons are the same in both cases because of the binary search properties of the random treap. \\

We can also say that if we want to sort a sequence, we can insert every element in an empty tree, and then read the tree in a prefix order, this operation will take have a complexity of $O(n\log n)$, the same as quicksort.

\section{Implementation}

This project has been implemented in Python, with 2 main classes: \verb|Treap| and \verb|TreapNode|, that represent respectively the treap and a node of the treap. \\

The \verb|TreapNode| doesn't have a lot of interesting things to talk about, as it has only getters and setters to its attributes (left child, right child, parent, key and priority).
In the other hand, the \verb|Treap| class includes every implementation of every algorithm
useful to implement the random treap. Left and Right rotation routines have been implemented, as the find method too (just a BST search), but there's nothing special about them, nevertheless we'll check out more closely the \verb|insert| and \verb|delete| code.

The \verb|insert| method has been implemented in 2 parts. A treap has to add the node as it was just
a simple binary search tree, and then has to apply some rotations so that it satisfies also the heap
property.

\begin{lstlisting}
  # BST step
  attach = self.find(elem, False)

  # node creation with random priority
  node = TreapNode(elem, random.random())

  # in case it's an empty tree
  if attach == None:
      self.root = node
      return

  node.setParent(attach)
  if elem <= attach.getBkey():
      attach.setLeft(node)
  else:
      attach.setRight(node)

  # Heap step
  while node.getParent() != None and node.getHkey() > node.getParent().getHkey():
      if node.isRightChild():
          self.leftRotation(node)
      else:
          self.rightRotation(node)
\end{lstlisting}

The \verb|delete| method cannot be the same as the BSL, as it could break some priorities
if not done well. So, for the chosen node, we will have to move it down the tree by rotations until
it's a leaf, so we can delete it without problems. If we don't want to break the priorities, we'll do a
rotation with the child who has the more priority (for example, if priority of left child $= 0.5$ and right child $=0.4$, we'll chose the left child, so he'll be on top of the current right child and it won't break the priorities).

\begin{lstlisting}
  node = self.find(elem)

  if node.getBkey() != elem:
      return

  # moves node down til' it's a leaf
  while node.getLeft() != None or node.getRight() != None:
      left = -1 if node.getLeft() == None else node.getLeft().getHkey()
      right = -1 if node.getRight() == None else node.getRight().getHkey()

      if left > right:
          self.rightRotation(node.getLeft())
      else:
          self.leftRotation(node.getRight())

  # deletes node
  if node.isRightChild():
      node.getParent().setRight(None)
  else:
      node.getParent().setLeft(None)
\end{lstlisting}
\section{Results}

Lemma 8.6 states: Let $T$ be a random treap for a set $S$ of size $n$. For an element $x \in S$ having rank $k$,

$$ E[\textrm{depth}(x)] = H_k + H_{n-k+1} -1 $$

Being $H_n$ the $n$-th harmonic number. \\

For the tests I've taken every integer in the range $[1,100]$ (inclusive), so we can know
that the element $x$ is has the rank $x$ in the set. Then, I've created $100000$ different
treaps, each one inserting the elements of the set in a random order (after a shuffle for example).

In expectance, the results clearly prove that the property stated in the lemma is true, as it's seen in the graph with the results:

\scalebox{0.7}{%% Creator: Matplotlib, PGF backend
%%
%% To include the figure in your LaTeX document, write
%%   \input{<filename>.pgf}
%%
%% Make sure the required packages are loaded in your preamble
%%   \usepackage{pgf}
%%
%% Figures using additional raster images can only be included by \input if
%% they are in the same directory as the main LaTeX file. For loading figures
%% from other directories you can use the `import` package
%%   \usepackage{import}
%% and then include the figures with
%%   \import{<path to file>}{<filename>.pgf}
%%
%% Matplotlib used the following preamble
%%   \usepackage{fontspec}
%%   \setmainfont{Times New Roman}
%%   \setsansfont{Verdana}
%%   \setmonofont{Courier New}
%%
\begingroup%
\makeatletter%
\begin{pgfpicture}%
\pgfpathrectangle{\pgfpointorigin}{\pgfqpoint{6.420000in}{5.390000in}}%
\pgfusepath{use as bounding box, clip}%
\begin{pgfscope}%
\pgfsetbuttcap%
\pgfsetmiterjoin%
\definecolor{currentfill}{rgb}{1.000000,1.000000,1.000000}%
\pgfsetfillcolor{currentfill}%
\pgfsetlinewidth{0.000000pt}%
\definecolor{currentstroke}{rgb}{1.000000,1.000000,1.000000}%
\pgfsetstrokecolor{currentstroke}%
\pgfsetdash{}{0pt}%
\pgfpathmoveto{\pgfqpoint{0.000000in}{0.000000in}}%
\pgfpathlineto{\pgfqpoint{6.420000in}{0.000000in}}%
\pgfpathlineto{\pgfqpoint{6.420000in}{5.390000in}}%
\pgfpathlineto{\pgfqpoint{0.000000in}{5.390000in}}%
\pgfpathclose%
\pgfusepath{fill}%
\end{pgfscope}%
\begin{pgfscope}%
\pgfsetbuttcap%
\pgfsetmiterjoin%
\definecolor{currentfill}{rgb}{1.000000,1.000000,1.000000}%
\pgfsetfillcolor{currentfill}%
\pgfsetlinewidth{0.000000pt}%
\definecolor{currentstroke}{rgb}{0.000000,0.000000,0.000000}%
\pgfsetstrokecolor{currentstroke}%
\pgfsetstrokeopacity{0.000000}%
\pgfsetdash{}{0pt}%
\pgfpathmoveto{\pgfqpoint{0.802500in}{0.592900in}}%
\pgfpathlineto{\pgfqpoint{5.778000in}{0.592900in}}%
\pgfpathlineto{\pgfqpoint{5.778000in}{4.743200in}}%
\pgfpathlineto{\pgfqpoint{0.802500in}{4.743200in}}%
\pgfpathclose%
\pgfusepath{fill}%
\end{pgfscope}%
\begin{pgfscope}%
\pgfsetbuttcap%
\pgfsetroundjoin%
\definecolor{currentfill}{rgb}{0.000000,0.000000,0.000000}%
\pgfsetfillcolor{currentfill}%
\pgfsetlinewidth{0.803000pt}%
\definecolor{currentstroke}{rgb}{0.000000,0.000000,0.000000}%
\pgfsetstrokecolor{currentstroke}%
\pgfsetdash{}{0pt}%
\pgfsys@defobject{currentmarker}{\pgfqpoint{0.000000in}{-0.048611in}}{\pgfqpoint{0.000000in}{0.000000in}}{%
\pgfpathmoveto{\pgfqpoint{0.000000in}{0.000000in}}%
\pgfpathlineto{\pgfqpoint{0.000000in}{-0.048611in}}%
\pgfusepath{stroke,fill}%
}%
\begin{pgfscope}%
\pgfsys@transformshift{0.982970in}{0.592900in}%
\pgfsys@useobject{currentmarker}{}%
\end{pgfscope}%
\end{pgfscope}%
\begin{pgfscope}%
\pgftext[x=0.982970in,y=0.495678in,,top]{\sffamily\fontsize{10.000000}{12.000000}\selectfont 0}%
\end{pgfscope}%
\begin{pgfscope}%
\pgfsetbuttcap%
\pgfsetroundjoin%
\definecolor{currentfill}{rgb}{0.000000,0.000000,0.000000}%
\pgfsetfillcolor{currentfill}%
\pgfsetlinewidth{0.803000pt}%
\definecolor{currentstroke}{rgb}{0.000000,0.000000,0.000000}%
\pgfsetstrokecolor{currentstroke}%
\pgfsetdash{}{0pt}%
\pgfsys@defobject{currentmarker}{\pgfqpoint{0.000000in}{-0.048611in}}{\pgfqpoint{0.000000in}{0.000000in}}{%
\pgfpathmoveto{\pgfqpoint{0.000000in}{0.000000in}}%
\pgfpathlineto{\pgfqpoint{0.000000in}{-0.048611in}}%
\pgfusepath{stroke,fill}%
}%
\begin{pgfscope}%
\pgfsys@transformshift{1.896744in}{0.592900in}%
\pgfsys@useobject{currentmarker}{}%
\end{pgfscope}%
\end{pgfscope}%
\begin{pgfscope}%
\pgftext[x=1.896744in,y=0.495678in,,top]{\sffamily\fontsize{10.000000}{12.000000}\selectfont 20}%
\end{pgfscope}%
\begin{pgfscope}%
\pgfsetbuttcap%
\pgfsetroundjoin%
\definecolor{currentfill}{rgb}{0.000000,0.000000,0.000000}%
\pgfsetfillcolor{currentfill}%
\pgfsetlinewidth{0.803000pt}%
\definecolor{currentstroke}{rgb}{0.000000,0.000000,0.000000}%
\pgfsetstrokecolor{currentstroke}%
\pgfsetdash{}{0pt}%
\pgfsys@defobject{currentmarker}{\pgfqpoint{0.000000in}{-0.048611in}}{\pgfqpoint{0.000000in}{0.000000in}}{%
\pgfpathmoveto{\pgfqpoint{0.000000in}{0.000000in}}%
\pgfpathlineto{\pgfqpoint{0.000000in}{-0.048611in}}%
\pgfusepath{stroke,fill}%
}%
\begin{pgfscope}%
\pgfsys@transformshift{2.810519in}{0.592900in}%
\pgfsys@useobject{currentmarker}{}%
\end{pgfscope}%
\end{pgfscope}%
\begin{pgfscope}%
\pgftext[x=2.810519in,y=0.495678in,,top]{\sffamily\fontsize{10.000000}{12.000000}\selectfont 40}%
\end{pgfscope}%
\begin{pgfscope}%
\pgfsetbuttcap%
\pgfsetroundjoin%
\definecolor{currentfill}{rgb}{0.000000,0.000000,0.000000}%
\pgfsetfillcolor{currentfill}%
\pgfsetlinewidth{0.803000pt}%
\definecolor{currentstroke}{rgb}{0.000000,0.000000,0.000000}%
\pgfsetstrokecolor{currentstroke}%
\pgfsetdash{}{0pt}%
\pgfsys@defobject{currentmarker}{\pgfqpoint{0.000000in}{-0.048611in}}{\pgfqpoint{0.000000in}{0.000000in}}{%
\pgfpathmoveto{\pgfqpoint{0.000000in}{0.000000in}}%
\pgfpathlineto{\pgfqpoint{0.000000in}{-0.048611in}}%
\pgfusepath{stroke,fill}%
}%
\begin{pgfscope}%
\pgfsys@transformshift{3.724293in}{0.592900in}%
\pgfsys@useobject{currentmarker}{}%
\end{pgfscope}%
\end{pgfscope}%
\begin{pgfscope}%
\pgftext[x=3.724293in,y=0.495678in,,top]{\sffamily\fontsize{10.000000}{12.000000}\selectfont 60}%
\end{pgfscope}%
\begin{pgfscope}%
\pgfsetbuttcap%
\pgfsetroundjoin%
\definecolor{currentfill}{rgb}{0.000000,0.000000,0.000000}%
\pgfsetfillcolor{currentfill}%
\pgfsetlinewidth{0.803000pt}%
\definecolor{currentstroke}{rgb}{0.000000,0.000000,0.000000}%
\pgfsetstrokecolor{currentstroke}%
\pgfsetdash{}{0pt}%
\pgfsys@defobject{currentmarker}{\pgfqpoint{0.000000in}{-0.048611in}}{\pgfqpoint{0.000000in}{0.000000in}}{%
\pgfpathmoveto{\pgfqpoint{0.000000in}{0.000000in}}%
\pgfpathlineto{\pgfqpoint{0.000000in}{-0.048611in}}%
\pgfusepath{stroke,fill}%
}%
\begin{pgfscope}%
\pgfsys@transformshift{4.638067in}{0.592900in}%
\pgfsys@useobject{currentmarker}{}%
\end{pgfscope}%
\end{pgfscope}%
\begin{pgfscope}%
\pgftext[x=4.638067in,y=0.495678in,,top]{\sffamily\fontsize{10.000000}{12.000000}\selectfont 80}%
\end{pgfscope}%
\begin{pgfscope}%
\pgfsetbuttcap%
\pgfsetroundjoin%
\definecolor{currentfill}{rgb}{0.000000,0.000000,0.000000}%
\pgfsetfillcolor{currentfill}%
\pgfsetlinewidth{0.803000pt}%
\definecolor{currentstroke}{rgb}{0.000000,0.000000,0.000000}%
\pgfsetstrokecolor{currentstroke}%
\pgfsetdash{}{0pt}%
\pgfsys@defobject{currentmarker}{\pgfqpoint{0.000000in}{-0.048611in}}{\pgfqpoint{0.000000in}{0.000000in}}{%
\pgfpathmoveto{\pgfqpoint{0.000000in}{0.000000in}}%
\pgfpathlineto{\pgfqpoint{0.000000in}{-0.048611in}}%
\pgfusepath{stroke,fill}%
}%
\begin{pgfscope}%
\pgfsys@transformshift{5.551841in}{0.592900in}%
\pgfsys@useobject{currentmarker}{}%
\end{pgfscope}%
\end{pgfscope}%
\begin{pgfscope}%
\pgftext[x=5.551841in,y=0.495678in,,top]{\sffamily\fontsize{10.000000}{12.000000}\selectfont 100}%
\end{pgfscope}%
\begin{pgfscope}%
\pgfsetbuttcap%
\pgfsetroundjoin%
\definecolor{currentfill}{rgb}{0.000000,0.000000,0.000000}%
\pgfsetfillcolor{currentfill}%
\pgfsetlinewidth{0.803000pt}%
\definecolor{currentstroke}{rgb}{0.000000,0.000000,0.000000}%
\pgfsetstrokecolor{currentstroke}%
\pgfsetdash{}{0pt}%
\pgfsys@defobject{currentmarker}{\pgfqpoint{-0.048611in}{0.000000in}}{\pgfqpoint{0.000000in}{0.000000in}}{%
\pgfpathmoveto{\pgfqpoint{0.000000in}{0.000000in}}%
\pgfpathlineto{\pgfqpoint{-0.048611in}{0.000000in}}%
\pgfusepath{stroke,fill}%
}%
\begin{pgfscope}%
\pgfsys@transformshift{0.802500in}{1.206356in}%
\pgfsys@useobject{currentmarker}{}%
\end{pgfscope}%
\end{pgfscope}%
\begin{pgfscope}%
\pgftext[x=0.478159in,y=1.153594in,left,base]{\sffamily\fontsize{10.000000}{12.000000}\selectfont 5.5}%
\end{pgfscope}%
\begin{pgfscope}%
\pgfsetbuttcap%
\pgfsetroundjoin%
\definecolor{currentfill}{rgb}{0.000000,0.000000,0.000000}%
\pgfsetfillcolor{currentfill}%
\pgfsetlinewidth{0.803000pt}%
\definecolor{currentstroke}{rgb}{0.000000,0.000000,0.000000}%
\pgfsetstrokecolor{currentstroke}%
\pgfsetdash{}{0pt}%
\pgfsys@defobject{currentmarker}{\pgfqpoint{-0.048611in}{0.000000in}}{\pgfqpoint{0.000000in}{0.000000in}}{%
\pgfpathmoveto{\pgfqpoint{0.000000in}{0.000000in}}%
\pgfpathlineto{\pgfqpoint{-0.048611in}{0.000000in}}%
\pgfusepath{stroke,fill}%
}%
\begin{pgfscope}%
\pgfsys@transformshift{0.802500in}{1.869120in}%
\pgfsys@useobject{currentmarker}{}%
\end{pgfscope}%
\end{pgfscope}%
\begin{pgfscope}%
\pgftext[x=0.478159in,y=1.816359in,left,base]{\sffamily\fontsize{10.000000}{12.000000}\selectfont 6.0}%
\end{pgfscope}%
\begin{pgfscope}%
\pgfsetbuttcap%
\pgfsetroundjoin%
\definecolor{currentfill}{rgb}{0.000000,0.000000,0.000000}%
\pgfsetfillcolor{currentfill}%
\pgfsetlinewidth{0.803000pt}%
\definecolor{currentstroke}{rgb}{0.000000,0.000000,0.000000}%
\pgfsetstrokecolor{currentstroke}%
\pgfsetdash{}{0pt}%
\pgfsys@defobject{currentmarker}{\pgfqpoint{-0.048611in}{0.000000in}}{\pgfqpoint{0.000000in}{0.000000in}}{%
\pgfpathmoveto{\pgfqpoint{0.000000in}{0.000000in}}%
\pgfpathlineto{\pgfqpoint{-0.048611in}{0.000000in}}%
\pgfusepath{stroke,fill}%
}%
\begin{pgfscope}%
\pgfsys@transformshift{0.802500in}{2.531885in}%
\pgfsys@useobject{currentmarker}{}%
\end{pgfscope}%
\end{pgfscope}%
\begin{pgfscope}%
\pgftext[x=0.478159in,y=2.479123in,left,base]{\sffamily\fontsize{10.000000}{12.000000}\selectfont 6.5}%
\end{pgfscope}%
\begin{pgfscope}%
\pgfsetbuttcap%
\pgfsetroundjoin%
\definecolor{currentfill}{rgb}{0.000000,0.000000,0.000000}%
\pgfsetfillcolor{currentfill}%
\pgfsetlinewidth{0.803000pt}%
\definecolor{currentstroke}{rgb}{0.000000,0.000000,0.000000}%
\pgfsetstrokecolor{currentstroke}%
\pgfsetdash{}{0pt}%
\pgfsys@defobject{currentmarker}{\pgfqpoint{-0.048611in}{0.000000in}}{\pgfqpoint{0.000000in}{0.000000in}}{%
\pgfpathmoveto{\pgfqpoint{0.000000in}{0.000000in}}%
\pgfpathlineto{\pgfqpoint{-0.048611in}{0.000000in}}%
\pgfusepath{stroke,fill}%
}%
\begin{pgfscope}%
\pgfsys@transformshift{0.802500in}{3.194650in}%
\pgfsys@useobject{currentmarker}{}%
\end{pgfscope}%
\end{pgfscope}%
\begin{pgfscope}%
\pgftext[x=0.478159in,y=3.141888in,left,base]{\sffamily\fontsize{10.000000}{12.000000}\selectfont 7.0}%
\end{pgfscope}%
\begin{pgfscope}%
\pgfsetbuttcap%
\pgfsetroundjoin%
\definecolor{currentfill}{rgb}{0.000000,0.000000,0.000000}%
\pgfsetfillcolor{currentfill}%
\pgfsetlinewidth{0.803000pt}%
\definecolor{currentstroke}{rgb}{0.000000,0.000000,0.000000}%
\pgfsetstrokecolor{currentstroke}%
\pgfsetdash{}{0pt}%
\pgfsys@defobject{currentmarker}{\pgfqpoint{-0.048611in}{0.000000in}}{\pgfqpoint{0.000000in}{0.000000in}}{%
\pgfpathmoveto{\pgfqpoint{0.000000in}{0.000000in}}%
\pgfpathlineto{\pgfqpoint{-0.048611in}{0.000000in}}%
\pgfusepath{stroke,fill}%
}%
\begin{pgfscope}%
\pgfsys@transformshift{0.802500in}{3.857414in}%
\pgfsys@useobject{currentmarker}{}%
\end{pgfscope}%
\end{pgfscope}%
\begin{pgfscope}%
\pgftext[x=0.478159in,y=3.804653in,left,base]{\sffamily\fontsize{10.000000}{12.000000}\selectfont 7.5}%
\end{pgfscope}%
\begin{pgfscope}%
\pgfsetbuttcap%
\pgfsetroundjoin%
\definecolor{currentfill}{rgb}{0.000000,0.000000,0.000000}%
\pgfsetfillcolor{currentfill}%
\pgfsetlinewidth{0.803000pt}%
\definecolor{currentstroke}{rgb}{0.000000,0.000000,0.000000}%
\pgfsetstrokecolor{currentstroke}%
\pgfsetdash{}{0pt}%
\pgfsys@defobject{currentmarker}{\pgfqpoint{-0.048611in}{0.000000in}}{\pgfqpoint{0.000000in}{0.000000in}}{%
\pgfpathmoveto{\pgfqpoint{0.000000in}{0.000000in}}%
\pgfpathlineto{\pgfqpoint{-0.048611in}{0.000000in}}%
\pgfusepath{stroke,fill}%
}%
\begin{pgfscope}%
\pgfsys@transformshift{0.802500in}{4.520179in}%
\pgfsys@useobject{currentmarker}{}%
\end{pgfscope}%
\end{pgfscope}%
\begin{pgfscope}%
\pgftext[x=0.478159in,y=4.467417in,left,base]{\sffamily\fontsize{10.000000}{12.000000}\selectfont 8.0}%
\end{pgfscope}%
\begin{pgfscope}%
\pgfpathrectangle{\pgfqpoint{0.802500in}{0.592900in}}{\pgfqpoint{4.975500in}{4.150300in}} %
\pgfusepath{clip}%
\pgfsetrectcap%
\pgfsetroundjoin%
\pgfsetlinewidth{1.505625pt}%
\definecolor{currentstroke}{rgb}{0.121569,0.466667,0.705882}%
\pgfsetstrokecolor{currentstroke}%
\pgfsetdash{}{0pt}%
\pgfpathmoveto{\pgfqpoint{1.028659in}{0.781550in}}%
\pgfpathlineto{\pgfqpoint{1.074348in}{1.433101in}}%
\pgfpathlineto{\pgfqpoint{1.120037in}{1.865502in}}%
\pgfpathlineto{\pgfqpoint{1.165725in}{2.186651in}}%
\pgfpathlineto{\pgfqpoint{1.211414in}{2.438594in}}%
\pgfpathlineto{\pgfqpoint{1.257103in}{2.649923in}}%
\pgfpathlineto{\pgfqpoint{1.302791in}{2.825145in}}%
\pgfpathlineto{\pgfqpoint{1.348480in}{2.953059in}}%
\pgfpathlineto{\pgfqpoint{1.394169in}{3.102857in}}%
\pgfpathlineto{\pgfqpoint{1.439857in}{3.216958in}}%
\pgfpathlineto{\pgfqpoint{1.485546in}{3.317805in}}%
\pgfpathlineto{\pgfqpoint{1.531235in}{3.412090in}}%
\pgfpathlineto{\pgfqpoint{1.576924in}{3.514473in}}%
\pgfpathlineto{\pgfqpoint{1.622612in}{3.560244in}}%
\pgfpathlineto{\pgfqpoint{1.668301in}{3.645462in}}%
\pgfpathlineto{\pgfqpoint{1.713990in}{3.719400in}}%
\pgfpathlineto{\pgfqpoint{1.759678in}{3.793245in}}%
\pgfpathlineto{\pgfqpoint{1.805367in}{3.855095in}}%
\pgfpathlineto{\pgfqpoint{1.851056in}{3.920655in}}%
\pgfpathlineto{\pgfqpoint{1.896744in}{3.950042in}}%
\pgfpathlineto{\pgfqpoint{1.942433in}{4.006364in}}%
\pgfpathlineto{\pgfqpoint{1.988122in}{4.048516in}}%
\pgfpathlineto{\pgfqpoint{2.033811in}{4.098462in}}%
\pgfpathlineto{\pgfqpoint{2.079499in}{4.136491in}}%
\pgfpathlineto{\pgfqpoint{2.125188in}{4.170929in}}%
\pgfpathlineto{\pgfqpoint{2.170877in}{4.203643in}}%
\pgfpathlineto{\pgfqpoint{2.216565in}{4.224308in}}%
\pgfpathlineto{\pgfqpoint{2.262254in}{4.266075in}}%
\pgfpathlineto{\pgfqpoint{2.307943in}{4.290968in}}%
\pgfpathlineto{\pgfqpoint{2.353632in}{4.311766in}}%
\pgfpathlineto{\pgfqpoint{2.399320in}{4.341484in}}%
\pgfpathlineto{\pgfqpoint{2.445009in}{4.358464in}}%
\pgfpathlineto{\pgfqpoint{2.490698in}{4.376942in}}%
\pgfpathlineto{\pgfqpoint{2.536386in}{4.406356in}}%
\pgfpathlineto{\pgfqpoint{2.582075in}{4.416814in}}%
\pgfpathlineto{\pgfqpoint{2.627764in}{4.437864in}}%
\pgfpathlineto{\pgfqpoint{2.673452in}{4.453465in}}%
\pgfpathlineto{\pgfqpoint{2.719141in}{4.461577in}}%
\pgfpathlineto{\pgfqpoint{2.764830in}{4.473242in}}%
\pgfpathlineto{\pgfqpoint{2.810519in}{4.487598in}}%
\pgfpathlineto{\pgfqpoint{2.856207in}{4.488804in}}%
\pgfpathlineto{\pgfqpoint{2.901896in}{4.494954in}}%
\pgfpathlineto{\pgfqpoint{2.947585in}{4.517223in}}%
\pgfpathlineto{\pgfqpoint{2.993273in}{4.524354in}}%
\pgfpathlineto{\pgfqpoint{3.038962in}{4.538047in}}%
\pgfpathlineto{\pgfqpoint{3.084651in}{4.542965in}}%
\pgfpathlineto{\pgfqpoint{3.130340in}{4.551515in}}%
\pgfpathlineto{\pgfqpoint{3.176028in}{4.552257in}}%
\pgfpathlineto{\pgfqpoint{3.221717in}{4.550892in}}%
\pgfpathlineto{\pgfqpoint{3.267406in}{4.554550in}}%
\pgfpathlineto{\pgfqpoint{3.313094in}{4.548996in}}%
\pgfpathlineto{\pgfqpoint{3.358783in}{4.539863in}}%
\pgfpathlineto{\pgfqpoint{3.404472in}{4.544224in}}%
\pgfpathlineto{\pgfqpoint{3.450160in}{4.548095in}}%
\pgfpathlineto{\pgfqpoint{3.495849in}{4.539147in}}%
\pgfpathlineto{\pgfqpoint{3.541538in}{4.532202in}}%
\pgfpathlineto{\pgfqpoint{3.587227in}{4.517223in}}%
\pgfpathlineto{\pgfqpoint{3.632915in}{4.520736in}}%
\pgfpathlineto{\pgfqpoint{3.678604in}{4.506566in}}%
\pgfpathlineto{\pgfqpoint{3.724293in}{4.486643in}}%
\pgfpathlineto{\pgfqpoint{3.769981in}{4.482812in}}%
\pgfpathlineto{\pgfqpoint{3.815670in}{4.478862in}}%
\pgfpathlineto{\pgfqpoint{3.861359in}{4.466535in}}%
\pgfpathlineto{\pgfqpoint{3.907048in}{4.450138in}}%
\pgfpathlineto{\pgfqpoint{3.952736in}{4.437307in}}%
\pgfpathlineto{\pgfqpoint{3.998425in}{4.420062in}}%
\pgfpathlineto{\pgfqpoint{4.044114in}{4.398654in}}%
\pgfpathlineto{\pgfqpoint{4.089802in}{4.406475in}}%
\pgfpathlineto{\pgfqpoint{4.135491in}{4.376094in}}%
\pgfpathlineto{\pgfqpoint{4.181180in}{4.349424in}}%
\pgfpathlineto{\pgfqpoint{4.226868in}{4.313158in}}%
\pgfpathlineto{\pgfqpoint{4.272557in}{4.289179in}}%
\pgfpathlineto{\pgfqpoint{4.318246in}{4.260905in}}%
\pgfpathlineto{\pgfqpoint{4.363935in}{4.235986in}}%
\pgfpathlineto{\pgfqpoint{4.409623in}{4.208242in}}%
\pgfpathlineto{\pgfqpoint{4.455312in}{4.172546in}}%
\pgfpathlineto{\pgfqpoint{4.501001in}{4.132289in}}%
\pgfpathlineto{\pgfqpoint{4.546689in}{4.091158in}}%
\pgfpathlineto{\pgfqpoint{4.592378in}{4.055727in}}%
\pgfpathlineto{\pgfqpoint{4.638067in}{4.013774in}}%
\pgfpathlineto{\pgfqpoint{4.683756in}{3.967579in}}%
\pgfpathlineto{\pgfqpoint{4.729444in}{3.911085in}}%
\pgfpathlineto{\pgfqpoint{4.775133in}{3.867515in}}%
\pgfpathlineto{\pgfqpoint{4.820822in}{3.816389in}}%
\pgfpathlineto{\pgfqpoint{4.866510in}{3.752989in}}%
\pgfpathlineto{\pgfqpoint{4.912199in}{3.670594in}}%
\pgfpathlineto{\pgfqpoint{4.957888in}{3.590055in}}%
\pgfpathlineto{\pgfqpoint{5.003576in}{3.508296in}}%
\pgfpathlineto{\pgfqpoint{5.049265in}{3.424709in}}%
\pgfpathlineto{\pgfqpoint{5.094954in}{3.331444in}}%
\pgfpathlineto{\pgfqpoint{5.140643in}{3.234124in}}%
\pgfpathlineto{\pgfqpoint{5.186331in}{3.122130in}}%
\pgfpathlineto{\pgfqpoint{5.232020in}{2.986078in}}%
\pgfpathlineto{\pgfqpoint{5.277709in}{2.830248in}}%
\pgfpathlineto{\pgfqpoint{5.323397in}{2.645205in}}%
\pgfpathlineto{\pgfqpoint{5.369086in}{2.447383in}}%
\pgfpathlineto{\pgfqpoint{5.414775in}{2.192549in}}%
\pgfpathlineto{\pgfqpoint{5.460463in}{1.879645in}}%
\pgfpathlineto{\pgfqpoint{5.506152in}{1.438098in}}%
\pgfpathlineto{\pgfqpoint{5.551841in}{0.794341in}}%
\pgfusepath{stroke}%
\end{pgfscope}%
\begin{pgfscope}%
\pgfpathrectangle{\pgfqpoint{0.802500in}{0.592900in}}{\pgfqpoint{4.975500in}{4.150300in}} %
\pgfusepath{clip}%
\pgfsetrectcap%
\pgfsetroundjoin%
\pgfsetlinewidth{1.505625pt}%
\definecolor{currentstroke}{rgb}{1.000000,0.498039,0.054902}%
\pgfsetstrokecolor{currentstroke}%
\pgfsetdash{}{0pt}%
\pgfpathmoveto{\pgfqpoint{1.028659in}{0.791965in}}%
\pgfpathlineto{\pgfqpoint{1.074348in}{1.441475in}}%
\pgfpathlineto{\pgfqpoint{1.120037in}{1.869929in}}%
\pgfpathlineto{\pgfqpoint{1.165725in}{2.187785in}}%
\pgfpathlineto{\pgfqpoint{1.211414in}{2.439226in}}%
\pgfpathlineto{\pgfqpoint{1.257103in}{2.646340in}}%
\pgfpathlineto{\pgfqpoint{1.302791in}{2.821748in}}%
\pgfpathlineto{\pgfqpoint{1.348480in}{2.973338in}}%
\pgfpathlineto{\pgfqpoint{1.394169in}{3.106366in}}%
\pgfpathlineto{\pgfqpoint{1.439857in}{3.224511in}}%
\pgfpathlineto{\pgfqpoint{1.485546in}{3.330447in}}%
\pgfpathlineto{\pgfqpoint{1.531235in}{3.426180in}}%
\pgfpathlineto{\pgfqpoint{1.576924in}{3.513250in}}%
\pgfpathlineto{\pgfqpoint{1.622612in}{3.592868in}}%
\pgfpathlineto{\pgfqpoint{1.668301in}{3.666001in}}%
\pgfpathlineto{\pgfqpoint{1.713990in}{3.733433in}}%
\pgfpathlineto{\pgfqpoint{1.759678in}{3.795811in}}%
\pgfpathlineto{\pgfqpoint{1.805367in}{3.853672in}}%
\pgfpathlineto{\pgfqpoint{1.851056in}{3.907466in}}%
\pgfpathlineto{\pgfqpoint{1.896744in}{3.957577in}}%
\pgfpathlineto{\pgfqpoint{1.942433in}{4.004333in}}%
\pgfpathlineto{\pgfqpoint{1.988122in}{4.048016in}}%
\pgfpathlineto{\pgfqpoint{2.033811in}{4.088868in}}%
\pgfpathlineto{\pgfqpoint{2.079499in}{4.127105in}}%
\pgfpathlineto{\pgfqpoint{2.125188in}{4.162911in}}%
\pgfpathlineto{\pgfqpoint{2.170877in}{4.196452in}}%
\pgfpathlineto{\pgfqpoint{2.216565in}{4.227872in}}%
\pgfpathlineto{\pgfqpoint{2.262254in}{4.257300in}}%
\pgfpathlineto{\pgfqpoint{2.307943in}{4.284850in}}%
\pgfpathlineto{\pgfqpoint{2.353632in}{4.310624in}}%
\pgfpathlineto{\pgfqpoint{2.399320in}{4.334714in}}%
\pgfpathlineto{\pgfqpoint{2.445009in}{4.357200in}}%
\pgfpathlineto{\pgfqpoint{2.490698in}{4.378157in}}%
\pgfpathlineto{\pgfqpoint{2.536386in}{4.397650in}}%
\pgfpathlineto{\pgfqpoint{2.582075in}{4.415739in}}%
\pgfpathlineto{\pgfqpoint{2.627764in}{4.432475in}}%
\pgfpathlineto{\pgfqpoint{2.673452in}{4.447907in}}%
\pgfpathlineto{\pgfqpoint{2.719141in}{4.462078in}}%
\pgfpathlineto{\pgfqpoint{2.764830in}{4.475026in}}%
\pgfpathlineto{\pgfqpoint{2.810519in}{4.486785in}}%
\pgfpathlineto{\pgfqpoint{2.856207in}{4.497385in}}%
\pgfpathlineto{\pgfqpoint{2.901896in}{4.506853in}}%
\pgfpathlineto{\pgfqpoint{2.947585in}{4.515213in}}%
\pgfpathlineto{\pgfqpoint{2.993273in}{4.522484in}}%
\pgfpathlineto{\pgfqpoint{3.038962in}{4.528686in}}%
\pgfpathlineto{\pgfqpoint{3.084651in}{4.533831in}}%
\pgfpathlineto{\pgfqpoint{3.130340in}{4.537933in}}%
\pgfpathlineto{\pgfqpoint{3.176028in}{4.541002in}}%
\pgfpathlineto{\pgfqpoint{3.221717in}{4.543043in}}%
\pgfpathlineto{\pgfqpoint{3.267406in}{4.544063in}}%
\pgfpathlineto{\pgfqpoint{3.313094in}{4.544063in}}%
\pgfpathlineto{\pgfqpoint{3.358783in}{4.543043in}}%
\pgfpathlineto{\pgfqpoint{3.404472in}{4.541002in}}%
\pgfpathlineto{\pgfqpoint{3.450160in}{4.537933in}}%
\pgfpathlineto{\pgfqpoint{3.495849in}{4.533831in}}%
\pgfpathlineto{\pgfqpoint{3.541538in}{4.528686in}}%
\pgfpathlineto{\pgfqpoint{3.587227in}{4.522484in}}%
\pgfpathlineto{\pgfqpoint{3.632915in}{4.515213in}}%
\pgfpathlineto{\pgfqpoint{3.678604in}{4.506853in}}%
\pgfpathlineto{\pgfqpoint{3.724293in}{4.497385in}}%
\pgfpathlineto{\pgfqpoint{3.769981in}{4.486785in}}%
\pgfpathlineto{\pgfqpoint{3.815670in}{4.475026in}}%
\pgfpathlineto{\pgfqpoint{3.861359in}{4.462078in}}%
\pgfpathlineto{\pgfqpoint{3.907048in}{4.447907in}}%
\pgfpathlineto{\pgfqpoint{3.952736in}{4.432475in}}%
\pgfpathlineto{\pgfqpoint{3.998425in}{4.415739in}}%
\pgfpathlineto{\pgfqpoint{4.044114in}{4.397650in}}%
\pgfpathlineto{\pgfqpoint{4.089802in}{4.378157in}}%
\pgfpathlineto{\pgfqpoint{4.135491in}{4.357200in}}%
\pgfpathlineto{\pgfqpoint{4.181180in}{4.334714in}}%
\pgfpathlineto{\pgfqpoint{4.226868in}{4.310624in}}%
\pgfpathlineto{\pgfqpoint{4.272557in}{4.284850in}}%
\pgfpathlineto{\pgfqpoint{4.318246in}{4.257300in}}%
\pgfpathlineto{\pgfqpoint{4.363935in}{4.227872in}}%
\pgfpathlineto{\pgfqpoint{4.409623in}{4.196452in}}%
\pgfpathlineto{\pgfqpoint{4.455312in}{4.162911in}}%
\pgfpathlineto{\pgfqpoint{4.501001in}{4.127105in}}%
\pgfpathlineto{\pgfqpoint{4.546689in}{4.088868in}}%
\pgfpathlineto{\pgfqpoint{4.592378in}{4.048016in}}%
\pgfpathlineto{\pgfqpoint{4.638067in}{4.004333in}}%
\pgfpathlineto{\pgfqpoint{4.683756in}{3.957577in}}%
\pgfpathlineto{\pgfqpoint{4.729444in}{3.907466in}}%
\pgfpathlineto{\pgfqpoint{4.775133in}{3.853672in}}%
\pgfpathlineto{\pgfqpoint{4.820822in}{3.795811in}}%
\pgfpathlineto{\pgfqpoint{4.866510in}{3.733433in}}%
\pgfpathlineto{\pgfqpoint{4.912199in}{3.666001in}}%
\pgfpathlineto{\pgfqpoint{4.957888in}{3.592868in}}%
\pgfpathlineto{\pgfqpoint{5.003576in}{3.513250in}}%
\pgfpathlineto{\pgfqpoint{5.049265in}{3.426180in}}%
\pgfpathlineto{\pgfqpoint{5.094954in}{3.330447in}}%
\pgfpathlineto{\pgfqpoint{5.140643in}{3.224511in}}%
\pgfpathlineto{\pgfqpoint{5.186331in}{3.106366in}}%
\pgfpathlineto{\pgfqpoint{5.232020in}{2.973338in}}%
\pgfpathlineto{\pgfqpoint{5.277709in}{2.821748in}}%
\pgfpathlineto{\pgfqpoint{5.323397in}{2.646340in}}%
\pgfpathlineto{\pgfqpoint{5.369086in}{2.439226in}}%
\pgfpathlineto{\pgfqpoint{5.414775in}{2.187785in}}%
\pgfpathlineto{\pgfqpoint{5.460463in}{1.869929in}}%
\pgfpathlineto{\pgfqpoint{5.506152in}{1.441475in}}%
\pgfpathlineto{\pgfqpoint{5.551841in}{0.791965in}}%
\pgfusepath{stroke}%
\end{pgfscope}%
\begin{pgfscope}%
\pgfsetrectcap%
\pgfsetmiterjoin%
\pgfsetlinewidth{0.803000pt}%
\definecolor{currentstroke}{rgb}{0.000000,0.000000,0.000000}%
\pgfsetstrokecolor{currentstroke}%
\pgfsetdash{}{0pt}%
\pgfpathmoveto{\pgfqpoint{0.802500in}{0.592900in}}%
\pgfpathlineto{\pgfqpoint{0.802500in}{4.743200in}}%
\pgfusepath{stroke}%
\end{pgfscope}%
\begin{pgfscope}%
\pgfsetrectcap%
\pgfsetmiterjoin%
\pgfsetlinewidth{0.803000pt}%
\definecolor{currentstroke}{rgb}{0.000000,0.000000,0.000000}%
\pgfsetstrokecolor{currentstroke}%
\pgfsetdash{}{0pt}%
\pgfpathmoveto{\pgfqpoint{5.778000in}{0.592900in}}%
\pgfpathlineto{\pgfqpoint{5.778000in}{4.743200in}}%
\pgfusepath{stroke}%
\end{pgfscope}%
\begin{pgfscope}%
\pgfsetrectcap%
\pgfsetmiterjoin%
\pgfsetlinewidth{0.803000pt}%
\definecolor{currentstroke}{rgb}{0.000000,0.000000,0.000000}%
\pgfsetstrokecolor{currentstroke}%
\pgfsetdash{}{0pt}%
\pgfpathmoveto{\pgfqpoint{0.802500in}{0.592900in}}%
\pgfpathlineto{\pgfqpoint{5.778000in}{0.592900in}}%
\pgfusepath{stroke}%
\end{pgfscope}%
\begin{pgfscope}%
\pgfsetrectcap%
\pgfsetmiterjoin%
\pgfsetlinewidth{0.803000pt}%
\definecolor{currentstroke}{rgb}{0.000000,0.000000,0.000000}%
\pgfsetstrokecolor{currentstroke}%
\pgfsetdash{}{0pt}%
\pgfpathmoveto{\pgfqpoint{0.802500in}{4.743200in}}%
\pgfpathlineto{\pgfqpoint{5.778000in}{4.743200in}}%
\pgfusepath{stroke}%
\end{pgfscope}%
\begin{pgfscope}%
\pgfsetbuttcap%
\pgfsetmiterjoin%
\definecolor{currentfill}{rgb}{1.000000,1.000000,1.000000}%
\pgfsetfillcolor{currentfill}%
\pgfsetfillopacity{0.800000}%
\pgfsetlinewidth{1.003750pt}%
\definecolor{currentstroke}{rgb}{0.800000,0.800000,0.800000}%
\pgfsetstrokecolor{currentstroke}%
\pgfsetstrokeopacity{0.800000}%
\pgfsetdash{}{0pt}%
\pgfpathmoveto{\pgfqpoint{4.600686in}{4.226273in}}%
\pgfpathlineto{\pgfqpoint{5.680778in}{4.226273in}}%
\pgfpathquadraticcurveto{\pgfqpoint{5.708556in}{4.226273in}}{\pgfqpoint{5.708556in}{4.254051in}}%
\pgfpathlineto{\pgfqpoint{5.708556in}{4.645978in}}%
\pgfpathquadraticcurveto{\pgfqpoint{5.708556in}{4.673756in}}{\pgfqpoint{5.680778in}{4.673756in}}%
\pgfpathlineto{\pgfqpoint{4.600686in}{4.673756in}}%
\pgfpathquadraticcurveto{\pgfqpoint{4.572908in}{4.673756in}}{\pgfqpoint{4.572908in}{4.645978in}}%
\pgfpathlineto{\pgfqpoint{4.572908in}{4.254051in}}%
\pgfpathquadraticcurveto{\pgfqpoint{4.572908in}{4.226273in}}{\pgfqpoint{4.600686in}{4.226273in}}%
\pgfpathclose%
\pgfusepath{stroke,fill}%
\end{pgfscope}%
\begin{pgfscope}%
\pgfsetrectcap%
\pgfsetroundjoin%
\pgfsetlinewidth{1.505625pt}%
\definecolor{currentstroke}{rgb}{0.121569,0.466667,0.705882}%
\pgfsetstrokecolor{currentstroke}%
\pgfsetdash{}{0pt}%
\pgfpathmoveto{\pgfqpoint{4.628464in}{4.561288in}}%
\pgfpathlineto{\pgfqpoint{4.906242in}{4.561288in}}%
\pgfusepath{stroke}%
\end{pgfscope}%
\begin{pgfscope}%
\pgftext[x=5.017353in,y=4.512677in,left,base]{\sffamily\fontsize{10.000000}{12.000000}\selectfont Tests}%
\end{pgfscope}%
\begin{pgfscope}%
\pgfsetrectcap%
\pgfsetroundjoin%
\pgfsetlinewidth{1.505625pt}%
\definecolor{currentstroke}{rgb}{1.000000,0.498039,0.054902}%
\pgfsetstrokecolor{currentstroke}%
\pgfsetdash{}{0pt}%
\pgfpathmoveto{\pgfqpoint{4.628464in}{4.358380in}}%
\pgfpathlineto{\pgfqpoint{4.906242in}{4.358380in}}%
\pgfusepath{stroke}%
\end{pgfscope}%
\begin{pgfscope}%
\pgftext[x=5.017353in,y=4.309769in,left,base]{\sffamily\fontsize{10.000000}{12.000000}\selectfont Expected}%
\end{pgfscope}%
\end{pgfpicture}%
\makeatother%
\endgroup%
}

The average error is $0.05$, which is pretty little and can shows us the acceptance of the lemma 8.6.

\section{Conclusion}

In conclusion, the treaps solve with an elegant twist some drawbacks of traditional tree-based
data structures with the power of randomness, and in this document it has been proved that in expectance, random treaps will not have the problem of the worst-case scenario that have simple
BSTs, with the operations having \textbf{the same complexity} ($O(\log n)$).

\end{document}
